\documentclass[a4paper,12pt,fleqn,oneside]{article} 
\input{Setup/Preamble.tex}
\def\titlename{Analyse}

\begin{document}
\input{Setup/Forside.tex}
\tableofcontents
\newpage
\section{Analyse}
I dette dokument beskrives det analyse arbejde, samt de forskellige overvejelser der har været i løbet af arbejdsprocessen. De største overvejelser for projektet har omhandlet den grafiske brugeroverfalde og databasen. I de næste to afsnit beskrives de overvejelser der har været, fordele og ulemper, samt det endelige valg. 

\subsection{Applikation og Grafisk Brugeroverflade}
For at brugeren kan interagere med vores system, skal der være en grafisk brugeroverflade. En anden essentielt faktor er, hvilke operativsystemer skal applikationen kunne virke på. Hertil undersøgte vi flere systemer/platforme. (Vi undersøgte også browser baserede løsninger, men det blev forkastet efter vi valgte at systemet skulle være en applikation). Det var desuden også bestemt at det skulle være på dotNet platformen. Vi endte derved mellem to valg: WPF eller Xamarin applikation - her introduceres nogen af fordelene og ulemperne ved begge: \\\\
\textbf{Windows Presentation Foundation (WPF, Windows Only):} \\
Fordele:
\begin{itemize}
    \item Stor erfaring fra undervisning og mulighed for at hjælp fra faglærere. 
    \item MVVM Pattern - let at overholde lav kobling mellem GUI og business logik. 
\end{itemize}
Ulemper: 
\begin{itemize}
    \item WPF er et subsystem til .NET platformen, som bruges til at lave applikation og den grafiske brugeroverflade dertil. Dette betyder at den er Windows only. 
    \item Erfaring siger den har en længere læringskurve. 
\end{itemize}
\textbf{Xamarin (Mobile App, Cross Platform):} \\ 
Fordele:\\
\begin{itemize}
    \item Xamarin applikationer er 'native', dvs. de udarbejdes til specifikke platforme. Xamarin er platformsuafhængig og kan bruges til Windows, iOS og OS X. 
    \item MVVM Pattern - let at overholde lav kobling mellem GUI og business logik. 
\end{itemize}
Ulemper: \\
\begin{itemize}
    \item Ingen i gruppen har erfaring med Xamarin 
    \item Ingen undervisning i Xamarin, alt skal undersøges selv
\end{itemize}
Ud fra ovenstående argumenter, valgte vi at lave en WPF applikation. Der er begrænset tid i løbet af et semesterprojekt, så det at have mulighed for at få hjælp af en faglærer, samt løbende at få mere erfaring for WPF, er en kæmpe fordel. Hvis derimod projektet skulle blive kommercialiseret, ville det have været nødvendigt at lave en applikation til alle platforme, for at konkurrere med GoMores platformuafhængige applikation. 

\subsubsection{Fonts}
Normalt er det først i designfasen, hvor man udvælger og definerer, hvilke fonts man bruger til de forskellige vinduer. Det er dog ikke alle fonts, som er gratis eller indbygget i WPF. Der bruges derfor kun fonts, som er gratis og indbygget i WPF. 

\subsection{Database:}
I dette afsnit beskrives de overvejelser, der har været i forhold til kreationen af databasen. Heri undersøges både lagring af data i form af database udbyder, samt hvordan data traditionelt overføres mellem brugere. 
\subsubsection{Valg af kommunikationstype}
\textbf{Interval polling \& event queries}\\
En udfordring med en GUI som er tilsluttet en database i dens model-lag er hvor synkroniseret disse skal være. Der startes med at definere vores mål og ønsker og problematikkerne ved at opnå disse. \newline

Vi ønsker at brugeren ikke ser forældet data, og derved forsøger at leje biler der allerede er lejet, eller ser en liste af biler som ikke længere er tilgængelige.\newline

Den naive, og set fra et User Experience synspunkt nemmeste løsning, er at polle databasen oftere en brugeren kan forventes at tage beslutninger baseret på det info de ser i GUI'ens view, en gang i sekundet for eksempel. Vores applikation er dog lavet til at mange brugere skal køre hver deres instans af applikationen og siden vi ikke har mulighed for at distribuere vores database over flere servere ville dette hurtigt overvælde databasen.\newline

I stedet for vælges det at opdatere GUI'ens lokale data når visse handlinger tages. Disse kan være at skifte til et andet view, så når brugeren klikker på en bestemt bil eller bringer en liste af biler frem er disse just opdaterede.\newline

Et problem med denne løsning er at det forudsætter en aktiv bruger, hvilket applikationen ikke kræver. Det er fuldt muligt for en bruger at have applikationen stående inaktiv længe og komme tilbage timer eller dage senere og så se forældet data. Vi løser dette ved at polle databasen, dog med en meget større interval, og med en timer der nulstilles hver gang der sendes en forspørgsel på databasen.\newline

Disse 2 løsninger sammen gør at GUI'ens lokale data aldrig er væsentligt forældet.

Vi har valgt en relationel database, da NOSQL er optimeret mere mod store mængder data af varierende type og mindre mod garantier for at data er fejlfri. Siden vi arbejder med relativt små mængder data som involvere personlig info er det vigtigere at data er garanteret for at være korrekt.
Kilde: https://blog.pandorafms.org/nosql-vs-sql-key-differences/
TODO(Should this be here?)




%Bare skriv analyse direkte i dette dokument, ikke brug for subfiles 

\printbibliography

\end{document}
