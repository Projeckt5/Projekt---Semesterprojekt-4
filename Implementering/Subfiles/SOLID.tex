\documentclass[Implementering/Implementering_main.tex]{subfiles}

\begin{document}
\section{SOLID}

SOLID er et akronym for fem vigtige design principper, systemet er designet sådan at implementeringen overholder disse fem principper så godt som muligt.

\subsubsection{Single Responsibility Principle}
Single respoinsibility principle: systemets klasser bør have ansvar over en ting. 

\subsubsection{Open/closed princilple}
Open/closed princilple: Systemets entiteter bør laves sådan at der er mulighed for viderebygning men ikke for direkte ændring, dette gøres f.eks ved nedarvning mellem klasser.

\subsubsection{Liskov substitution principle}
Liskov substitution principle: Hvis et system har subtyper så kan en subtype af et objekt ”substitute” for den objekt type.

\subsubsection{Interface segregation principle}
Interface segregation principle: Dele af systemet bør ikke være afhængige af metoder den ikke benytter. 

\subsubsection{Dependency inversion principle}
Dependency inversion principle: Høj niveau moduler bør ikke være afhængige af lav niveau moduler, begge bør afhænge af abstraktioner. 


\end{document}