\documentclass[a4paper,12pt,fleqn,oneside]{article} 
\input{Setup/Preamble.tex}

\def\titlename{Proces}

\begin{document}

%===============FORSIDE======================
\input{Setup/Forside.tex}
\subfile{ProcesDokument/Versionshistorik/Versionshistorik.tex}\newpage
\tableofcontents \newpage 

\section{Forord}
Dette dokument er skrevet af PRJ4 gruppe 5, der en projektgruppe bestående af udelukkende IKT-studerende fra Aarhus Universitet. Dette dokument er udarbejdet som en beskrivelse af den proces, gruppen har været igennem under hele semesteret og har til formål, at være en opsummering og sammenbringelse af procesrelateret information for, hvem end det må interessere. En forudsætning for at forstå dette dokument, at i hvert fald Projekt Introduktionen og noget af Kravspecifikationen er læst forud for dette. Derudover fungere dette dokument som en uddybning af Proces. \\\\
Der er desuden taget inspiration fra et tidligere projekt, PRJ3 Gruppe 7, da samme udviklingsværktøjer tages i brug. I det forrige projekt blev Scrum også brugt som et værktøj til at styre udviklingsprocessen. 

\section{Indledning}

\section{Gruppedannelse}

\section{Udviklingsforløb}
\subsection{Introduktion}
\subsection{Scrum}
\subsubsection{Sprints}

\subsubsection{Rollerne i Scrum}
\textbf{ScrumMaster} \\
\textbf{ProductOwner} \\
\textbf{Team Member} \\
\textbf{Scrum Relaterede møder} 
\textbf{Redskaber} \\ 

\subsection{Erfaringer med Scrum}
\subsubsection{Sprints}

\subsection{Udviklingsværktøjer fra 2. og 3. Semester}
\subsubsection{SysML}

\section{Projektledelse}
From chaos to self-organising (Se Simon brown, https://www.infoq.com/presentations/The-Frustrated-Architect) 

\section{Arbejdsfordeling}

\section{Planlægning}
\subsection{Iterativ Planlægning}

\section{Møder}
\subsection{Vejledermøder}
\subsection{Gruppemøder}
\subsection{Review}

\section{Konflikthåndtering}

\end{document}