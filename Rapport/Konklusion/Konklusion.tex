\documentclass[Rapport/Rapport_main.tex]{subfiles}
\begin{document}
\section{Konklusion}
I dette afsnit konkluderes der på Projektformuleringen og kravene til projektet, og det beskrives kort, hvad der er lykkedes, og hvad der ikke er. Udover resultaterne konkluderes der på udviklingsprocessen og de erfaringer, der er opnået i løbet af denne.\\\\Der er implementeret en biludlejningsapplikation kaldet CarnGo. Den har en grafisk brugergrænseflade, som giver brugerne mulighed for at leje og udleje biler. Brugeren skal først logge ind med en e-mail og password på 'Login'-siden. Disse gemmes i en relationel database, og hvis man ikke er registreret i forvejen har man mulighed for at navigere til en 'Register'-side og oprette sig. Når brugeren er logget ind kan der tilgås en side med personlige informationer, hvor man kan indtaste navn og adresse eller uploade et billede. Der skelnes ikke mellem, om man er lejer eller udlejer ved oprettelsen, men denne oplysning kan angives her.\\\\Som udlejer kan der tilgås en yderligere side, hvor man kan oprette en bilprofil ved at indtaste relevante oplysninger og uploade et billede af bilen, der herefter gemmes i databasen. \textbf{(Muligvis ikke implementeret?)} Ved at navigere til 'Find Car'-siden kan man se de biler, der er gemt i databasen, og som dermed kan lejes. Det er muligt at søge mellem bilerne ud fra lokation, bilmærke, antal sæder, afhentnings- og leveringstidspunkt eller en kombination af disse. Når brugeren har fundet en bil, som det ønskes at leje, bliver man ved tryk på en 'Rent'-knap ledt til en 'Rent Car'-side. Her er der angivet yderligere oplysninger om bilen såsom registreringsnummer, pris, brændstof og en kort beskrivelse af bilen. Der vælges nu udlejningsperiode, og der skrives en besked til udlejer. Brugere kan se deres beskeder ved at trykke på notifikationer i headerbaren. Herfra kan der navigeres til en 'Message'-side, hvor alle ens indgående og udgående beskeder er vist, og her kan udlejer vælge at godkende eller afvise en anmodning fra lejer. Endelig kan brugeren logge ud af applikationen, hvilket fører tilbage til 'Login'-siden. \\\\Det kan konkluderes, at store dele af applikationen er implementeret og virker efter hensigten. Databasen gemmer informationer om brugere og biler, det er muligt at leje biler og brugere kan kommunikere indbyrdes gennem beskedsystemet. Der har netop været fokus på at få den grundlæggende funktionalitet implementeret, og det fremgår, at dette er lykkedes. Konsekvensen har dog været, at visse ting måtte nedprioriteres. F.eks. har der ikke været fokus på sikkerhed/kryptering, eller anvendelsen af lokalitetstjenester. Der mangler således en form for regulering, da forhold såsom betaling og afhentningsted må aftales indbyrdes af brugerne. CarnGo er lavet som en Desktop Application i WPF, og det er ikke lykkedes at gøre den tilgængelig på flere platforme. Hvis applikationen skulle have været platformsuafhængig havde det været mere oplagt at anvende Xamarin, men dette blev fravalgt grundet manglende erfaring og undervisning i teknologien.\\\\Endelig kan det konkluderes, at der er anvendt en iterativ udviklingsproces. Den har taget udgangspunkt i Scrum, og dette har bl.a. medført, at der løbende er blevet ændret i kravspecifikation og arkitektur. Overordnet har tidsestimeringen i udviklingsprocessen været succesfuld, da den vigtigste funktionalitet er implementeret i CarnGo. Det kunne dog godt optimeres, og den lave Coverage indikerer, at der godt kunne bruges mere tid på Unit Tests og integrationstests.

\end{document}