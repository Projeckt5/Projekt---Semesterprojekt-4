\documentclass[Rapport/Rapport_main.tex]{subfiles}
\begin{document}
\section{Resume}
I dette projekt anvendes en Scrum baseret udviklingsproces til at implementere og teste en biludlejningsapplikation. Der er fokuseret på at udarbejde en grafisk brugergrænseflade samt en database, som understøtter udlejningsprocessen. Systemet giver privatpersoner mulighed for at oprette en brugerprofil som henholdsvis lejer eller udlejer. Udlejere kan registrere biler i systemet, mens lejere kan søge efter tilgængelige biler ud fra udvalgte kriterier. Når man har oprettet en profil som lejer eller udlejer kan man logge ind i applikationen. Her er det muligt at redigere eller fjerne sin brugerprofil og bilprofiler. Når lejer har valgt en bil kan der sendes en anmodning om leje til den pågældende udlejer. Denne får en notifikation og kan herefter vælge at godkende eller afvise anmodningen. Ved godkendelse er handelen gennemført og lejer notificeres herom. \\\\Informationer vedrørende brugere og biler gemmes i en relationel database, som hostes af Microsoft Azure. Det er valgt at anvende frameworket WPF, hvilket begrænser applikationen til at fungere på en PC med Windows OS som den eneste platform. Desuden er det fravalgt at gøre brug af lokalitetstjenester, data kryptering og forsikringer. Store dele af systemet er implementeret og unit testet. Systemet er herefter integrationstestet med udgangspunkt i en ''Sandwich?'' strategi og en udarbejdet integrationsplan. Endelig er der udført automatiserede såvel som manuelle accepttests. Ikke alle krav opfyldt, yderlige arbejde påkrævet? To be continued

\section{Abstract}
In this project a Scrum based development process is used to implement and test a car rental application. The focus is on making a graphical user interface and a database that support the car rental process. The system allows individuals to register a user profile as either renter or car owner. Car owners can register cars in the system, while renters are able to search for available cars based on some set of chosen criteria. When the user profile has been created, the renter or car owner can log into the application. This allows them to edit or remove the user profile or car profiles. Once the renter has decided on a car, it is possible to send a request to the car owner in question. The car owner receives a notification and has to decide whether to accept or reject the request. If it is accepted the transaction is completed and the renter is notified.\\\\Information regarding users and cars is stored in a relational database, which is hosted by Microsoft Azure. It was chosen to use the framework WPF and since it only supports a single platform, the application is limited to work on a PC with Windows OS. Furthermore, it was decided that no location services, data encryption or ensurance policies were to be used in the project. Large parts of the system have been implemented and unit tested. This was followed by integration tests in which a ''Sandwich?'' strategy and an integration plan were utilized. Finally, automated as well as manual acceptance tests were performed. Not all requirements met, further work necessary? To be continued 
\end{document}