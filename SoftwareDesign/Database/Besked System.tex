\documentclass[SoftwareDesign/SoftwareDesign_main.tex]{subfiles}
\begin{document}
\section{Kommunikation system}
En vigtig ting i CarnGo applikationen er, at der kan kommunikeres mellem en lejer og udlejer. Til at få dette til at lykkedes skulle der laves et kommunikations system. Ideen var at kommunikation skulle foregå igennem en database, hvor en lejer kunne lægge en besked op i databasen til udlejer om leje af en af udlejers biler(tidsinterval samt tekstbesked). Udlejer ville igennem databasen notificeres om at en besked er modtaget, hvorved han læser beskeden og kan accepterer eller afvise. Når beskeden er accepteret eller afvist vil lejer få besked ved at beskeden i databasen er ændret, og lejer bliver notificeret om dette.

\subsection{Entitetter i databasen}
For selve kommunikationssystem er der nogle entities i databasen der er vigtige for at få det hele til at virke. Disse entities er Messages, som er selve beskeden i databasen med tekstbeskeden, samt email (primary key for user) for både lejer og udlejer samt sender og modtager. Der er en Confirmationstatus variabel for at lejer kan se om leje af bil er godkendt, afvist eller venter på bekræftelse. Der er også en besked type variabel for at vise om beskeden er fra lejer til udlejer eller omvendt. Der er også tilføjet en boolean for om en besked er set, som ville blive vigtig for at få notifikationer til at virke. Det er også vigtigt med en reference til bil entitetten, da både lejer og udlejer skal vide, hvilken bil korrespondancen omhandler. En besked har en reference til en junction tabel entitet mellem en bruger og en besked. Der kan nemlig være flere beskeder tilknyttet samme bruger(enten lejer eller udlejer) og for en besked kan der være flere brugere(en udlejer og en lejer. Det er vigtigt for en bruger at have en reference til sendte men også modtagede beskeder. 
\\
PossibleToRentDays og DaysThatIsRented er begge entitetter, der ikke er en del af selve kommunikationssystemet, men bliver påvirket af handlinger foretaget under korrespondancen. Hver bil i databasen har reference til en PossibleToRentDays og DaysThatIsRented, hvor PossibleToRentDays er de mulige dage bilen kan lejes og DaysThatIsRented er hvilke af disse dage bilen er udlejet. Når lejer sender besked til udlejer, så vil den valgte tidsperiode for leje af bil lægges ind i databasen( som DaysThatIsRented). Dette vil sige at andre lejere ikke kan leje bilen i denne tidsperiode før udlejer har afvist lejeren, hvorved DaysThatIsRented vil slettes fra databasen.

\subsection{Håndtering i applikation}
Hele kommunikationen starter altid hos lejeren, der starter korrespondancen i applikationens SendRequestView ved at sende en anmodning om leje af bil til udlejer. Her vil DaysThatIsRented blive lagt ind i databasen for bilen med tidsperiode for udlejning og en besked med junktion tabeller til brugere vil blive lagt ind i databasen.
\\
I udlejers applikation var det meningen at udlejeren skulle notificeres om en ændring i applikationens Header bar. Dette blev dog ikke lavet, og vil forklares i afsnittet "Problemer med Notifikationer". I stedet vil notificeringen først dukke om i en pop op menu, når notifikations ikonet i header bar trykkes. I denne notifikation vil, der på udlejers side vises en knap til at accepterer lejers anmodning og ligeledes en knap til at afvise den. Notifikationen kan også åbnes ved at trykke på den(andre steder end de to knapper), hvorved hele tekstbeskeden vil ses. I tilfældet hvor udlejer afviser en anmodning vil DaysThatIsRented dage i frigives(slettes)på bilen i databasen og andre kan nu leje bilen i denne tidsperiode. I tilfældet hvor udlejer derimod bliver godkendt bliver der ikke ændret noget i databasen udover ConfirmationStatus, som ændres lige meget hvilket valg udlejer foretager. 
\\ 
I lejers applikation vil lejer nu kunne se om anmodningen er accepteret eller afvist ved at trykke på notifikationes ikonet i header baren. Igen vil han pga. problemer med notifikationer ikke kunne notificeres om ændringen automatisk uden at trykke på notifikations ikonet.
\subsection{Problemer med notifikationer}
Det var meningen, at både lejer og udlejer ville blive notificeret om at en besked var sendt til dem uden at en aktiv handling(tryk på notifikations ikon i header bar) fra brugeren skulle opdaterer headerbar med nye beskeder.
\\
Den første ide var at bruge Query notifikations, som var blevet nævnt under database undervisningen, hvor sql serveren vil være i stand til at udløse en trigger, når databasen bliver opdateret, hvorved applikationen vil notificeres om ændringer og eventuelt kan læse fra databasen og hermed UI. Det blev dog hurtigt klart at dette ikke var en mulighed, da den azure databases, som er den cloud database leverandør, der bruges i projektet ikke understøtter query notifications. 
\\
Den anden ide var at applikationen skulle polle databasen for ændringer i en funktion, der bliver udløst af en timer med et specifikt tidsinterval. Dette skulle foregå i en anden tråd, så UI tråden ikke blev påvirket. Det viste sig dog, at programmet crasher, når databasen bliver tilgået på flere tråde, hvor tråden notifikations polle tråden crasher og dermed programmet. Det var utroligt svært at debugge problemet og vi måtte til sidst give op på det. En mulig løsning på kunne være at bruge en unit-of-work design pattern. Der var dog ikke tid til at implementere denne design pattern.


\end{document}