\documentclass[Kravspecifikation/Kravspec_Main.tex]{subfiles}
\begin{document}

\section{Kvalitets Analyse}
I dette avsnittet vil kvalitets attributtene og overveielsene for CarNGo diskuteres og analyseres. Denne gjennomgangen vil følge software quality modellen, det vil gjennomgås hvilke kvaliteter som er vektlagt og hvilke som er ned prioritert med en begrunnelse.

\textbf{Functional suitability}\\
Det blir mye vekt lagt på at produktet skal kunne oppfylle de kravenene som har blitt spesifisert i kravspesifikasjonen.

\textbf{Performance efficiency}\\
Dette er et punkt som ikke har blitt lagt så mye vekt på i starten av utviklingen siden det skal først utvikles et «proof of consept» men effektiviteten av programvaren kan oppdateres senere i utviklings prossessen/etter initial launch

\textbf{Compatibility}\\
Dette punktet har ikke blitt prioritert siden programvaren skal ikke kommunisere med noen eksterne aktører i sin nåværende form.

\textbf{Usability}\\
Dette punktet har høy prioritet siden det er essensielt for å har et suksessfullt produkt i dette markedet at det skal være enkelt å bruke. Hvis produktet er vanskelig å bruke vil det også føre til en mindre kunde base.

\textbf{Reliability}\\
Det er viktig at er produkt som dette er stabilt siden det skal direkte fasilitere kommunikasjonen mellom kundene, hvis systemet skulle gå ned kunne dette forårsake problemer i utleinings prosess som ville føre til et tap av salg.

\textbf{Security}\\
Sikkerheten er et viktig punkt i denne programvaren siden den skal kunne betalinger mellom kundene og utleierne. Men sikkerheten skal veies mot brukervennligheten av produktet.

\textbf{Maintainabiliy}\\
Det er et mål å gjøre produktet så vedlikeholdbart som mulig fra starten av siden det skal lett kunne utvides for å kunne legge til ny funksjonalitet for å kunne holde følge med markedet men også siden vedlikeholdbarheten til programvare går ned over tid derfor er det best å ha et godt punkt å begynne fra.


\end{document}