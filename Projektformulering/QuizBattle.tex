\documentclass[a4paper,12pt,fleqn,oneside]{article} 
\input{Setup/Preamble.tex}
\def\titlename{Projektformulering}

% Den oprindelige projektformulering

\begin{document}
\input{Setup/Forside.tex}
\section{Indledning}
Målet for dette projekt er at udarbejde et Quiz spil. Formålet med dette er, at skabe en underholdende og udfordrende Quiz, hvor det er muligt for spillere at konkurrere indbyrdes over internettet. Det udvikles med henblik på anvendelse på forskellige platforme, herunder en mobilapplikation samt en webserver. Brugeren logger ind på den givne applikation, hvor der vælges, om der skal startes et nyt spil mod en tilfældig modstander, eller sendes en request til en specifik modstander. Herefter startes et spil mellem to brugere, som præsenteres for et antal kategorier. Den ene bruger vælger kategori og retten til at vælge kategori går på tur efter et vist antal spørgsmål. Brugerne henter et eller flere spørgsmål fra en database, og der vises fire svarmuligheder. Spillerne har et vist antal sekunder til at svare. Når tiden udløber vises svaret i applikationen efterfulgt af et nyt spørgsmål. Der vælges en ny kategori efter tre spørgsmål, og efter 12 spørgsmål er spillet slut og spillets resultat fremgår. Der er desuden mulighed for at tilgå sin account og se spilhistorik, scoreboard og spiller oplysninger.

\section{Problemformulering}
I dette projekt udvikles en mobilapplikation, som giver to personer mulighed for at konkurrere på paratviden. Applikationen er udformet som et Quiz spil, hvor brugerne besvarer spørgsmål inden for udvalgte kategorier. Der anvendes en database til at opbevare data, herunder spørgsmål, svar og brugerdata. 

\begin{figure}[H]
    \centering
    \includegraphics[width=\textwidth]{Projektformulering/graphic/1Skitse.jpg}
    \caption{Første Skitse af GUI}
    \label{fig:1_GUI}
\end{figure}

\begin{figure}[H]
    \centering
    \includegraphics[width=\textwidth]{Projektformulering/graphic/2Skitse.jpg}
    \caption{Anden Skitse af GUI}
    \label{fig:2_GUI}
\end{figure}

\section{MoSCoW}

\end{document}