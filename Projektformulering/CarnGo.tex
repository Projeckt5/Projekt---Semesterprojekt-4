\documentclass[a4paper,12pt,fleqn,oneside]{article} 
\input{Setup/Preamble.tex}
\def\titlename{Projektformulering}

% Den oprindelige projektformulering

\begin{document}
\input{Setup/Forside.tex}
\tableofcontents \newpage
\section{Produktmotivation}
Som studerende eller pensionist er det tit for dyrt eller upraktisk at have egen bil. Samtidig kan situationer opstå, hvor det ville være praktisk at have en bil til rådighed. Desværre er de nuværende udlejnings metoder ofte for upraktiske eller for uoverskuelige. Der er heller ikke en nem måde at leje fra privatpersoner, hvor der for begge parter stilles en garanti for sikkerhed. 

Privat leasing firmaet "GoMore" har allerede kommercialiseret selvsagte princip. Her kan en privat udlejer sætte sin bil til udleje i en given tidsperiode og angive en pris. GoMore har dog fået adskillige klager for deres aflysningspolitik i forhold til udlejerens mulighed for at aflyse et godkendt udleje. Det nævnes flere steder, at der er manglende tillid mellem udlejer og lejer, samt GoMore ikke har opsat en tryg ramme for deres udlejningsportal.\footnote{Mangler ref fra Trustpilot} Tillid og tryghed for lejer og udlejer skal således have højeste prioritet for dette produkt, samt en intuitiv og brugervenlig grænseflade.   

\section{Indledning}
Applikationen \textbf{CarnGo} gør op med besværligheden ved udlejnings-/leje-processen, og har til formål at simplificere leje-processen for almindelige borgere. Samtidig så vil den også simplificere udleje processen og gør det muligt for privat personer at udleje deres egen bil. Funktionaliteten skal samles, så det kan tilgås både fra en applikation på en smartphone, men også på nettet på en hjemmeside. Dette gøre brugeren i stand til at leje en bil, mens brugeren er på farten.

\subsection{Problemformulering}
I projektet udvikles en applikation, der hjælper privat personer med udlejning og leje af en bil. Ved leje af en bil, giver applikationen brugeren mulighed for at leje en bil, der opfylder brugerens krav og egne behov. Det kunne f.eks. være et maksimum beløb, eller en maksimum afstand hen til bilen. Ved udlejning af bil skal brugeren kunne opstille krav for, hvor lang tid bilen kan udlejes, og hvor meget en eventuel lejer skal betale for leje af bilen. For at sikre vilkår for både lejere og udlejere, skal en registrering/oprettelse foretages. Applikationen skal kunne tilgås på flere platforme, som PC og Smartphone.

\subsection{Systembeskrivelse}
Ud fra problemformuleringen kan der opstilles en skitse for, hvordan brugere ser og anvender systemet. Denne ses i figur \ref{fig:Systemskitse}.

\begin{figure}[H]
    \centering
    \includegraphics[width=\textwidth]{Projektformulering/graphic/SystemSkitsePRJ4.png}
    \caption{Systemskitse for CarnGo}
    \label{fig:Systemskitse}
\end{figure}

Systemet skal gennem brugergrænseflader på forskellige applikations platforme, forbedre leje-processen mellem en \textit{lejer} og udlejer, og samtidig gøre det nemt at søge og leje af en bil, der dækker lejers behov. For at sikrer begge parter skal de kunne registrere sig som brugere af systemet, der opbevarer informationer om parterne og aktioner de har foretaget. Samtidig skal det også foretage et tjek af personerne for at undgå snyd.

\end{document}